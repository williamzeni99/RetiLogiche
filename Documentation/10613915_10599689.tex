\documentclass{article}

% Language setting
% Replace `english' with e.g. `spanish' to change the document language
\usepackage[italian]{babel}

% Set page size and margins
% Replace `letterpaper' with`a4paper' for UK/EU standard size
\usepackage[letterpaper,top=2cm,bottom=2cm,left=3cm,right=3cm,marginparwidth=1.75cm]{geometry}

% Useful packages
\usepackage{graphicx}
\usepackage{amsmath}
\usepackage{graphicx}
\usepackage[colorlinks=true, allcolors=blue]{hyperref}
\usepackage{makeidx}

\title{Progetto di Reti Logiche}
\author{
  William Zeni\\
  \texttt{matricola 10613915}
  \and
  Cristina Urso\\
  \texttt{matricola 10599689}
}

\graphicspath{ {./graphics} }

\makeindex

\begin{document}
\maketitle

\renewcommand{\abstractname}{ }
\begin{abstract}
  \centering
  Progetto sostenuto presso il Politecnico di Milano, diretto dal professor Gianluca Palermo nell'anno 2020/21.
\end{abstract}

\begin{figure}[b]
  \centering
  \includegraphics[scale=0.4]{Logo_Politecnico_Milano.png}
\end{figure}


\pagebreak
\tableofcontents
\pagebreak


\section{Introduzione} %\index{Introduzione}
\subsection{Scopo del progetto} %\index{Introduzione! Scopo del progetto}
Write somenthing here

\subsection{Specifiche generali} %\index{Introduzione! Specifiche generali}
Write somenthing here

\subsection{Interfaccia del componente} %\index{Introduzione! Interfaccia del componente}
Write somenthing here

\subsection{Dati e Descrizione memoria} %\index{Introduzione! Dati e Descrizione memoria}
Write somenthing here

\section{Desing Pattern} %\index{Desing Pattern}
\subsection{Scelte Progettuali} %\index{Desing Pattern! Scelte Progettuali}
Write somenthing here

\subsubsection{START} %\index{Desing Pattern! Scelte Progettuali!START}
Lo stato di \texttt{START} è stato pensato come stato di attesa iniziale. Questo stato viene invocato in due situazioni differenti: se il segnale di \texttt{i\_rst} viene portato alto, oppure quando il segnale \texttt{i\_start} viene riportato basso dopo la computazione di un immagine. Lo stato \texttt{START} rimane tale fino a quando il segnale \texttt{i\_start} non venga posto alto. In quel momento lo stato successivo viene impostato \texttt{INIT}.

\subsubsection{INIT}%\index{Desing Pattern! Scelte Progettuali!INIT}
Lo stato \texttt{INIT} è uno stato di transizione nel quale il processore si assicura che i segnali siano inizializzati con i valori opportuni. Successivamente imposta lo stato prossimo a \texttt{ABILIT\_READ}.

\subsubsection{ABILIT\_READ}%\index{Desing Pattern! Scelte Progettuali! ABILIT\_READ}
Lo stato \texttt{ABILIT\_READ} è lo stato attraverso il quale abilitiamo la memoria alla sola lettura. Viene richiamato in momenti diversi del progetto ed in base allo stato chiamante, instrada lo stato prossimo a quello opportuno.

\subsubsection{ABILIT\_WRITE}%\index{Desing Pattern! Scelte Progettuali! ABILIT\_WRITE}
Lo stato \texttt{ABILIT\_WRITE} abilita la memoria alla lettura e alla scrittura. Viene invocato subito dopo aver computato il valore del nuovo pixel e in nessun altro momento. Instrada poi lo stato prossimo a \texttt{WRITE\_PIXEL}.

\subsubsection{WAIT\_MEM}%\index{Desing Pattern! Scelte Progettuali!WAIT\_MEM}
Lo stato \texttt{WAIT\_MEM} è uno stato centrale durante la gestione del flusso di dati. Sostanzialmente "spreca" un ciclo di clock. Questo ci assicura sia in caso di scrittura, sia in caso di lettura che i segnali in ingress e in uscita siano letti o scritti correttamente. Nel caso specifico alla quale ci rifacciamo, alcune chiamate a questo stato potevano essere evitate. Questa informazione è emersa durante lo stress test a cui il processore è stato sottoposto. Tuttavia, abbiamo preferito lasciarle per mantenere la stuttura del processore. Ciò, a nostro avviso, permette una maggior robustezza, sebbene una aumento nella latenza della computazione.

\subsubsection{GET\_RC}%\index{Desing Pattern! Scelte Progettuali! GET\_RC}
Lo stato \texttt{GET\_RC} è uno stato in preparazione al calcolo della dimensione dell'immagine e dei punti in cui bisognerà scrivere all'interno della memoria. Lo stato \texttt{GET\_RC} viene invocato dopo l'abilitazione della memoria alla lettura. Questo stato si occupa del recuperare i valori dalla memoria e aggiornare i segnali \texttt{n\_col} e \texttt{n\_row}.

\subsubsection{GET\_DIM}%\index{Desing Pattern! Scelte Progettuali! GET\_DIM}
Write somenthing here

\subsubsection{READ\_PIXEL}%\index{Desing Pattern! Scelte Progettuali! READ\_PIXEL}
Write somenthing here

\subsubsection{GET\_MINMAX}%\index{Desing Pattern! Scelte Progettuali! GET\_MINMAX}
Write somenthing here

\subsubsection{GET\_DELTA}%\index{Desing Pattern! Scelte Progettuali! GET\_DELTA}
Write somenthing here

\subsubsection{CALC\_SHIFT}%\index{Desing Pattern! Scelte Progettuali! CALC\_SHIFT}
Write somenthing here

\subsubsection{GET\_PIXEL}%\index{Desing Pattern! Scelte Progettuali!GET\_PIXEL}
Write somenthing here

\subsubsection{CALC\_NEWPIXEL}%\index{Desing Pattern! Scelte Progettuali!CALC\_NEWPIXEL}
Write somenthing here

\subsubsection{WRITE\_PIXEL}%\index{Desing Pattern! Scelte Progettuali!WRITE\_PIXEL}
Write somenthing here

\subsubsection{DONE}%\index{Desing Pattern! Scelte Progettuali!DONE}
Write somenthing here

\subsubsection{WAITINGPIC}%\index{Desing Pattern! Scelte Progettuali!WAITINGPIC}
Write somenthing here

\section{Risultati dei Test}%\index{Risultati dei Test}
Write somenthing here!

\section{Conclusioni}%\index{Conclusioni}
Write somenthing here

\end{document}
